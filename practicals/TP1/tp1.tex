
\documentclass[12pt]{article}
\usepackage{../../eplcrypto}
\usepackage{geometry} % see geometry.pdf on how to lay out the page. There's lots.
\geometry{a4paper} % or letter or a5paper or ... etc
% \geometry{landscape} % rotated page geometry
\usepackage[parfill]{parskip}
% See the ``Article customise'' template for come common customisations
\usepackage{amsmath}
\usepackage{graphicx}
\usepackage{amssymb}
\usepackage{algorithmic}
\usepackage{algorithm}
\title{Practical lesson 1}

%%% BEGIN DOCUMENT
\begin{document}

\maketitle

\textbf{Exercise 1:} Perfect secrecy\\
We define the following encryption scheme for messages, keys and ciphertexts in $\mathcal{M}=\mathcal{C}=\mathcal{K}=\mathbb{Z}_n$, where $\mathbb{Z}_n$ is essentially the integers in the inverval $[0,n)$,(in fact ($\mathbb{Z}_n$, +) forms a group).
\begin{itemize}
\item $Gen$ outputs a key $k \in \mathcal{K}$ selected uniformly at random
\item $Enc_k(m):=k+m$ mod $n$
\item $Dec_k(c):=c-k$ mod $n$
\end{itemize}
Suppose messages are drawn from $\mathcal{M}$ according to the binomial distribution. More precisely, $M ~ Bi(n-1,p)$ for some probability $p$ which means that $\forall m \in \mathcal{M}:Pr[M=m]=\binom{n-1}{m}p^m(1-p)^{n-1-m}$
\begin{enumerate}
\item \emph{Show that the encryption scheme above is perfectly secret}\\
Perfect secrecy:\begin{itemize}
\item Each key is used with equal probability (Gen takes uniformly at random)
\item For every plaintext x and ciphertext y there is a unique key k s.t. $Enc_k(x)=y$
\end{itemize}
\begin{equation*}
|\mathcal{K}|=|\mathcal{M}|=|\mathbb{Z}_n|
\end{equation*}
From condition 1 we obtain the following:
\begin{equation*}
\begin{aligned}
    Pr[ C=c \,|\,  M=m] &= Pr[K+M=c \, | \, M = m] \\
    &= Pr[K=c-m]\\
    &= \frac{1}{|\mathcal{K}|}
\end{aligned}
\end{equation*}
From condition 2:
\begin{equation*}
\begin{aligned}
    Pr[ M=m \,|\,  C=c] &= \frac{Pr[ C=c \,|\,  M=m] * Pr[M=m]}{Pr[C=c]} \\
    &= Pr[M=m]
\end{aligned}
\end{equation*}
Scheme is perfectly secret.
\item \emph{Evaluate $Pr[C=c]$ for every $c \in C$}
\begin{equation*}
\begin{aligned}
    Pr[C=c] &= \sum\limits_{m \in \mathcal{M}} Pr[C=c | M=m]*Pr[M=m] \\
    &= \frac{1}{|K|} \sum\limits_{m \in \mathcal{M}}  Pr[M=m]\\
    &= \frac{1}{|K|}
\end{aligned}
\end{equation*}
\item \emph{Evaluate $Pr[K=k|C=c]$ for every $k \in \mathcal{K}$ and $c \in \mathcal{C}$}
\begin{equation*}
\begin{aligned}
    Pr[ K=k \,|\,  C=c] &= Pr[C-M=k \, | \, C = c] \\
    &= Pr[c-M=k]\\
    &= Pr[M=c-k] \,(note:\,m=c-k)\\
    &= \binom{n-1}{m}p^m(1-p)^{n-1-m}
\end{aligned}
\end{equation*}
\end{enumerate}
\newpage

\textbf{Exercise 2:} Negligible functions\\
\begin{enumerate}
\item \emph{Let $f$ be a negligible function in $n$. Show that $g:\mapsto 1000*f(n)$ is negligible too.}\\
$f$ is negligible iff:\\
$\forall$ positive polynomial $p$, $\exists$N such that $\forall n\ge N$:\\
\begin{equation}
f(n) \le \frac{1}{p(n)}
\end{equation}
This definition also holds for the case in the question, $1000*f(n)$ is still negligible.
\item \emph{Show that the function $n\mapsto n^{-log(n)}$ is negligible in $n$}.\\
let $p(n)=a*n^k, a > 0$\\
f is negligible iff for all positive polynomial p, there exist an N such that for all $n\geq N$: $ f(n) \leq \frac{1}{p(n)}$.
\begin{equation*}
n^{-log(n)} \le \frac{1}{a*n^k} = a^{-1}* n^{-k}
\end{equation*}
\begin{equation*}
-log(n) \le -log(a) - k
\end{equation*}
\begin{equation*}
log(n) \ge log(a) + k
\end{equation*}
\begin{equation*}
n \ge e^{k+log(a)}
\end{equation*}
So for any polynomial, we can go to $n \ge e^{k+log(a)}$ and get an N that. after all the values n, the function will be negligible.\\
\end{enumerate}
\newpage



\textbf{Exercise 3:} Efficiency\\
Explain why the function that maps $n$ on a sequence of "1" of length $|\sqrt{n}|$ cannot be evaluated by an efficient algorithm in the size of the entry.
\begin{algorithm}
\begin{algorithmic}
    \REQUIRE $n \geq 0$
    \ENSURE A sequence of $\sqrt{n}$ ``$1$''
    \FOR{$i=0$ to $\lfloor\sqrt{n}\rfloor$}
        \STATE Print `1'
    \ENDFOR
\end{algorithmic}
\caption{example of algorithm}
\label{alg1}      
\end{algorithm}
Hint: see $n$ as a power of $2$.\\
An algorithm A is efficient if there exist a PPT p such that, the steps required are:
  \[ A(x) \leq p(|x|) \]
The input $n$ can be expressed under binary form as: \[n = 2^{|n|}\]
  Let's say that $k = |n|$. We know that the algorithm has to do at least $\sqrt{n}$ steps.
 \[\sqrt{n} = \sqrt{2^k} = 2^{\frac{k}{2}}\]
  Which is exponential, not polynomial.
\newpage




\textbf{Exercise 4:} Security model\\
Let $\negl$ denote a negligible function.
Remember that $\Pi \define \langle \Gen, \Enc, \Dec \rangle$ has \emph{indistinguishable
multiple encryption in the presence of eavesdroppers} if $\forall$
PPT $\A$, $\exists$ $\negl$:
  \[\Pr[\PrivKmult(n)=1]\leq\frac12+\negl(n) \,,\]
where $\PrivKmult(n)$ is defined as follows.
\begin{enumerate}
\item   $\A$ outputs $M_0=(m_0^1,\ldots,m_0^t),
M_1=(m_1^1,\ldots,m_1^t)$
\item Choose $k \leftarrow \G(1^n)$ and $b \leftarrow \{0,1\}$, and send
  $(\Enc_k(m_b^1),\ldots,\Enc_k(m_b^t))$ to $\A$
\item $\A$ outputs $b'$
\item Define $\PrivKmult(n) \define 1$ iff $b=b'$
\end{enumerate}

Also remember that $\Pi \define \langle \Gen, \Enc, \Dec \rangle$ has \emph{indistinguishable
encryption under a chosen-plaintext attack} if $\forall$ PPT $\A$,
$\exists$ $\negl$:
  \[\Pr[\PrivKcpa(n)=1]\leq\frac12+\negl(n) \,,\]
where $\PrivKcpa(n)$ is defined as follows.

\begin{enumerate}
  \item Choose $k\leftarrow \Gen(1^n)$
  \item \textbf{$\A$ is given oracle access to $\Enc_k(\cdot)$}
  \item $\A$ outputs $m_0, m_1 \in \M$
  \item Choose $b\pick \bset$ and send $\Enc_k(m_b)$ to $\A$
  \item \textbf{$\A$ is again given oracle access to $\Enc_k(\cdot)$}
  \item $\A$ outputs $b'$
  \item Define $\PrivKcpa(n) \define 1$ iff $b=b'$
\end{enumerate}

Define the concept of indistinguishable \emph{multiple} encryption under a chosen-plaintext attack.\\
See the book.
%"left-or-right oracle" $\textbf{LR}_{sk,b}$, that, on input a pair of equal-length messages $m_0,m_1$, computes the ciphertext $c \leftarrow \Enc_k (m_b)$ and returns c. \\
%\newpage
%\textbf{The LR-oracle experiment} $\PrivK_{A,\Pi}^{\text{LR-cpa}}$:
%\begin{enumerate}
%    \item Choose $k \leftarrow \Gen(1^n)$
%    \item A uniform bit $b \in \{0,1\}$ is chosen
%    \item $\A$ is given oracle access to $\textbf{LR}_{sk,b}(\cdot,\cdot)$
%    \item $\A$ outputs $M_0 = (m_0^1, \ldots, m_0^t)$, $M_1 = (m_1^1, \ldots, m_1^t)$
%    \item Choose $b \leftarrow \{0,1\}$, and send $(\Enc_k(m_b^1), \ldots, \Enc_k(m_b^t))$ to $\A$
%    \item $\A$ is again given oracle access to $\Enc_k(\cdot)$
%    \item $\A$ outputs $b'$
%    \item Define$\PrivK_{A,\Pi}^{\text{LR-cpa}}(n) \define 1$ iff $b = b'$
%  \end{enumerate}






\end{document}