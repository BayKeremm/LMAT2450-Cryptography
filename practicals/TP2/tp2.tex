
\documentclass[12pt]{article}
\usepackage{../../eplcrypto}
\usepackage{geometry} % see geometry.pdf on how to lay out the page. There's lots.
\geometry{a4paper} % or letter or a5paper or ... etc
% \geometry{landscape} % rotated page geometry
\usepackage[parfill]{parskip}
% See the ``Article customise'' template for come common customisations
\usepackage{amsmath}
\usepackage{graphicx}
\usepackage{amssymb}
\usepackage{algorithmic}
\usepackage{algorithm}
\title{Practical lesson 2}

%%% BEGIN DOCUMENT
\begin{document}

\maketitle

\textbf{Exercise 1:} An attack\\
Let $F$ be a PRP. Consider the mode of operation in which a uniform value $IV \in \{0,1\}^n$ is chosen, and the $i$-th ciphertext block $c_i$ is computed as\\
 \begin{equation*}
 c_i \define F_k(IV+i+m_i)
 \end{equation*}
  where each $m_i \in \{0,1\}^n$ and the addition is performed modulo $2^n$. \\
 Show that this scheme is not EAV-secure.\\
 \textbf{Solution:}\\
 Select the following messages:
 \begin{equation*}
m_0 = (0\dots01, 0\dots0)
 \end{equation*} 
 \begin{equation*}
m_1 = (0\dots0, 0\dots0)
 \end{equation*} 
 \textbf{Reminder: Experiment} $\PrivKeav(n)$
\begin{enumerate}
\item $\mathcal{A}(1^n)$ outputs $m_0,m_1$ of identical lengths
\item pick $k \leftarrow Gen(1^n)$ and  $b \leftarrow \{0,1\}$ and send $c \leftarrow Enc_k(m_b)$ to $\mathcal{A}$
 \item $\mathcal{A}(c)$ outputs $b'$
 \item Define $Priv_{\mathcal{A},\prod}^{eav}(n):=1$ iff $b=b'$
\end{enumerate}
$\Pi \define \langle \Gen, \Enc, \Dec \rangle $ has \emph{indistinguishable} encryptions in the presence of eavesdroppers if $\forall$ PPT $\mathcal{A}$, $\exists$ negl. $\epsilon$:
\begin{equation*}
Pr[\PrivKeav(n)] = \frac{1}{2} + \epsilon(n)
\end{equation*}


\begin{align*}
c = 
\left\{
    \begin {aligned}
         & (Enc_k(IV+0+1), Enc_k(IV+1+0) \quad & for \,b=0 \\
         & (Enc_k(IV+0+0), Enc_k(IV+1+0) \quad & for \,b=1
    \end{aligned}
\right.
\end{align*}

\begin{align*}
\text{output}\, b= 
\left\{
    \begin {aligned}
         & 0 \quad & \text{if} \,c_0=c_1 \\
         & 1 \quad & \text{else}
     \end{aligned}
\right.
\end{align*}

So:
\begin{equation*}
Pr[\PrivKeav(n)] = 1
\end{equation*}
Which is not negligible, so the scheme is not EAV-secure.


\textbf{Exercise 2:} Reduction\\
Let $\Pi \define \langle \Gen,\Enc,\Dec\rangle$ be an encryption scheme having
indistinguishable encryption under a chosen plaintext attack. Suppose we
define a new scheme $\Pi' \define \langle \Gen',\Enc',\Dec'\rangle$ as follows.
\smallskip
\begin{itemize}
  \item $\Gen' \define \Gen$
  \item $\Enc_k'(m) \define \Enc_k(m)||1$ (i.e. a `1' bit is appended to the ciphertext)
  \item $\Dec_k'(c) \define \Dec_k(c_1)$, where $c_1$ is obtained by discarding the last bit of $c$.
\end{itemize}
\begin{enumerate}
\item Is $\Pi'$ also a mult-EAV secure encryption scheme? Provide an attack or a reduction depending on your claim.\\
Adversary $\A$ attacking $\Pi$, using adversary $\A'$ attacking $\Pi'$, as a subroutine.\begin{itemize}
\item $\A'$ outputs two messages of length n, $m_0, m_1$
\item $\A$ takes $m_0, m_1$, and forwards it to $\O$
\item $\O$ having the key, and having selected a random $b \in \{0,1\}$ sends back to $\A$ the ciphertext $c=\Enc_k(m_b)$
\item $\A$ having received the ciphertext c, concatenates 1, as $c'=c||1$, and sends it to $\A'$
\item $\A'$ outputs $b'$, and $\A$ outputs the same $b'$
\item $\A$ wins against $\Pi$ if $b'=b$, which happens if $\A'$ wins against $\Pi'$, which is emulated by $\A$.
\end{itemize}
Therefore,
\begin{equation*}
Pr[\PrivKeav(n)=1] = Pr[\PrivK_{\A',\Pi'}^{eav}(n)=1]
\end{equation*}
Keeping in mind the theorem, if a scheme is EAV-secure for one message, it is EAV-secure for multi-message setting (See the reference book).
\item Answer the same question but considering the CPA security of $\Pi'$, assuming that $\Pi$ is CPA secure\\
The same reduction as above, but now with the CPA-experiment, $\PrivKcpa$
\end{enumerate}
\newpage



\textbf{Exercise 3:} PRG\\
Let $G$ be a PRG with expansion factor $l(n)>2n$. In each of the following cases, say whether $G'$ is necessarily a PRG. If yes give a proof, if not, show a counterexample.\\
\textbf{Reminder PRG:} A deterministic poly-time algorithm $G$ is a pseudorandom generator (PRG) only if, for any n,
\begin{itemize}
\item $\forall s \in \{0,1\}^n, G(s) \in \{0,1\}^{l(n)}$ and $l(n) > n$
\item $\forall$ PPT $D$, $\exists$ negligible function $\epsilon$ s.t.
\begin{equation*}
|Pr[D(r)=1]-Pr[D(G(s))=1]| \le \epsilon(n)
\end{equation*}
Where $r\leftarrow \{0,1\}^{l(n)}$ and $s\leftarrow \{0,1\}^n$\\
$s$ is called the seed, and funciton $l$ us called the expansion factor of $G$.
\end{itemize}
\begin{enumerate}
\item \emph{Define $G'(s)=G(s_1\dots s_{n/2})$, where $s=s_1\dots s_n \in \{0,1\}^n$}\\\\
%The distinguisher $\D$ selects two seeds:\\
%\begin{equation*}
%s_0 = (s\dots s, s_0^1,s_0^2,\dots,s_0^{n/2})
%\end{equation*}
%\begin{equation*}
%s_1 = (s\dots s, s_1^1,s_1^2,\dots,s_1^{n/2})
%\end{equation*}
%Where the first n/2 bits are the same of the two selected seeds $s_0, s_1$. Then the distinguisher $\D$ outputs 1 if the two seeds return the same $G(s)$ output. Note: by Kerckhoffs’ principle, the specification of $G$ is known to $\D$. Here we see that the distinguisher will be able to tell if it is dealing with a PRG or not, so this construction is not a PRG.
We have $l(n) > 2n$, so $|G'(s)| > \frac12*2n=n$, as required by any PRG!\\
Let $l'$ be the expansion factor of $G'$.  FIx a PPT algorithm $D$ and set
\begin{equation*}
\negl(n) \define |Pr[D(r)=1]-Pr[D(G'(s))=1]|
\end{equation*}
Where, $r \leftarrow \{0,1\}^{l'(n)}$ and $s \leftarrow \{0,1\}^n$.\\
By definition of $G'$, we have that
\begin{equation*}
Pr[D(G'(s))=1] = Pr[D(G(s'))=1]
\end{equation*}
Where $s' \leftarrow \{0,1\}^{n/2}$\\
Therefore, we arrive at the following:

\begin{equation*}
|Pr[D(r)=1] - Pr[D(G(s'))=1]| = \negl(n) = \negl'(n/2)
\end{equation*}
Where $\negl'(n) \define \negl(2n)$, since $\negl'$ must be negligible (from the question statement that G is negligible), we conclude that $\negl$ is negligible as well. So $G'$ is a PRG.
\\
\item \emph{$G'(s)=G(0^{|s|}||s)$ where $s \in \{0,1\}^n$, that is, we prepend $|s|$ '0' bits to s.}\\\\
No, let's define $H: \{0,1\}^n \rightarrow \{0,1\}^{4n}$ as a PRG \\
and let's think $G(s)=H(s_1\dots s_{n/2})$
Then,
\begin{equation*}
G'(s) = G(0^{|s|}||s) = H(0^{|s|})
\end{equation*}
The problem here is that now it looks like $G'$, defined in terms of H above, is only using $0^{|s|}$ as input, which is not uniformly distributed, as was required in the PRG definition above.

\item \emph{Define $G'(s) \define G(s) || G(s+1)$}

let's define $F: \{0,1\}^n \rightarrow \{0,1\}^{2n}$ as a PRG \\
and let's think $G(s)=F(s_1\dots s_{n-1})$. If the last bit of s is 0, then we have
\begin{equation*}
G'(s) \define G(s) || G(s+1) = F(s_1\dots s_{n-1})||F(s_1\dots s_{n-1})
\end{equation*}
So, now s and s+1 only differ in the last bit (since it was zero), this means that with probability 1/2 the two halves of the output of $G'$ are the same. This is not a PRG.





\end{enumerate}















\end{document}