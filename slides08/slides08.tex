\documentclass[12pt]{article}
\usepackage{../eplcrypto}
\usepackage{geometry} % see geometry.pdf on how to lay out the page. There's lots.
\geometry{a4paper} % or letter or a5paper or ... etc
% \geometry{landscape} % rotated page geometry
\usepackage[parfill]{parskip}
% See the ``Article customise'' template for come common customisations
\usepackage{amsmath}
\usepackage{graphicx}
\usepackage{amssymb}
\usepackage{algorithmic}
\usepackage{algorithm}
\title{Slides08}

%%% BEGIN DOCUMENT
\begin{document}

\maketitle
\tableofcontents
\newpage



\section{Message integrity revisited}
Goal: ensure integrity and origin of message. We had a solution with MACs, however, the same key was used to generate and check MACs. Anyone who can check a MAC can also forge one. This will not work if all participants do not trust each other(that is why we concluded in slides05 that MAC will work in small communities)

\section{Digital Signature}
Principle:
\begin{itemize}
	\item Bob generates a key $pk,sk$
	\item $sk$ is kept secret, and used to sign messages
	\item $pk$ is made public, and used to verify signatures
\end{itemize}

So:
\begin{itemize}
	\item Only Bob can produce signatures
	\item Anyone (who has Bob's public key) can verify that Bob's signature is authentic.
\end{itemize}

\subsection{Advantages over MAC}
Simpler key management:
\begin{itemize}
	\item Signature: With one key pair, Bob can send authenticates messages to as many users as he wants.
	\item MAC: typically, Bob will need one key per contact
\end{itemize}

Publicly verifiable:
\begin{itemize}
	\item Signature: If Alice receives a mesage signed by Bob, she knows that everyone else will also consider it authentic
	\item MAC: Bob could have sent a valid $\Mac_k(m)$ to Alice but an invalid $\Mac_{k'}(m)$ to Steve, so not publicly verifiable
\end{itemize}

Transferable
\begin{itemize}
	\item Signature: Alice can bring a signed message to a third party (a judge) and convince him that Bob signed the message.
	\item MAC:
	\begin{itemize}
		\item Alice would need to reveal the key to the third party
		\item Even if she does, Alice could have generated the MAC herself
	\end{itemize}
\end{itemize}

Non repudiable:
\begin{itemize}
	\item Bob cannot later deny that he signed the message
\end{itemize}

\subsection*{Then why would we use MACs? }



\begin{itemize}
	\item Because electronic signature is more expensive. Same as "symmetric" vs "asymmetric" 	encryption argument: 100-1000 times less efficient (for short messages at least)
	\item Because they are very useful to strengthen symmetric encryption: non-malleable / 		authenticated encryption.
\end{itemize}

\subsection{Definition of a signature scheme}
A signature scheme is a triple $\Pi \define \langle \Gen, \Sign, \Vrfy \rangle$
\begin{itemize}
	\item $\Gen$ probabilistically selects $(pk,sk) \leftarrow \Gen(1^n)$
	\item $\Sign$ provides $\sigma \leftarrow \Sign_{sk}(m)$
	\item $\Vrfy$ outputs a bit $b \define \Vrfy_{pk}(m,\sigma)$
\end{itemize}

s.t. $\forall n, (pk,sk)\leftarrow \Gen(1^n), \forall m:$
\begin{equation*}
\Vrfy_{pk}(m, \Sign_{sk}(m))=1
\end{equation*}
except with negligible probability. This can be defined for fixed as well as arbitrary length messages.
\newpage
\subsubsection{Usage}
\begin{itemize}
	\item Bob uses $\Gen$ to generate a key pair (pk,sk)
	\item Bob advertises $pk$
	\item When he wants to transmit $m$, Bob computes $\sigma \leftarrow \Sign_{sk}(m)$ and sends $(m,\sigma)$
	\item Receiver retrieves $pk$
	\item Receiver checks that $\Vrfy_{pk}(m,\sigma)=1$. This ensures that:
	\begin{itemize}
		\item $m$ originates from Bob
		\item m has not been modified
		\item sheesh
	\end{itemize}
\end{itemize}
Remarks:
\begin{itemize}
	\item It does not say when $m$ was emitted.
	\item Not that $m$ is not a replay
\end{itemize}
For these issues, specific measures have to be added.


\subsection{Definition of security for digital signature}
intuition: No adversary can produce a valid signature for a message that was not previously signed, even if he can obtain signatures of messages of his choice. This would be called a forgery.

\subsubsection{Experiment: $\Sigforge$}
Define the signature forgery experiment:
\begin{itemize}
	\item Choose $(pk,sk) \leftarrow \Gen(1^n)$
	\item $\A$ receives $pk$ and oracle access to $\Sign_{sk}(\cdot)$ for messages of his choice (denote $\Q$ the set of messages he queries and receives a signature)
	\item $\A$ outputs $(m, \sigma)$
	\item Define $\Sigforge(n) \define 1 \iff \Vrfy_{pk}(m,\sigma)=1$ and $m \notin \Q$
\end{itemize}
A signature scheme $\Pi \define \langle \Gen, \Sign, \Vrfy \rangle$ is \emph{existentially unforgeable under an adaptive chosen-message attack} (EUF-CMA) if $\forall$ PPT $\A$, $\exists$ negl. $\negl$:
\begin{equation*}
	Pr[\Sigforge(n)] \le \negl(n)
\end{equation*}

\section{Schnorr's Signature Scheme}
Non-interactive Schnorr can be turned into a signature scheme:
\begin{itemize}
	\item Hash message together with a
\end{itemize}

Definition:
\begin{itemize}
	\item $\Gen(1^n)$ runs $\G(1^n)$ to obtain $\langle \GG, q, g \rangle$, then picks $x \leftarrow \ZZ_q$, sets $h=g^x$ and returns $(pk,sk)=(h,x)$
	\item $\Sign_x(m)$ picks $r \leftarrow \ZZ_q$, sets $a=g^r, e=\H(a,m), f=r+e\cdot x$, and returns $(e,f)$
	\item $\Vrfy_h(m,(e,f))$ computes $a=g^f/h^e$ and checks if $\H(a,m)=e$ 
\end{itemize}
Observe: h is not included in the inputs of $\H$, not needed because $h$ is published before signing.

\subsection{Security}
Schnorr's signature scheme is EUF-CMA secure in the ROM assuming that the DL is hard with respect to $\G$.

Let:
\begin{itemize}
	\item $\A$ be an EUF-CMA adversary against Schnorr's signature making at most $t$ queries to the RO (all distinct) and winning with probability $\negl$
	\item $P^*$ be an adversary against the soundness of Schnorr's protocol
\end{itemize}
It can be shown that $P^*$ can win with probability $\negl/t$ + negl. If $\negl$ not-negligible, then:
\begin{itemize}
	\item The DL problem is not hard w.r.t. $\G$ or Schnorr's protocol is not sound
\end{itemize}

$P^*$ proceeds as follows:
\begin{itemize}
	\item Receives proof statement $h=g^x$, submit it as public key to $\A$
	\item Picks a random $j \leftarrow {1, \dots, t}$
	\item When $\A$ makes its $i$-th RO query query, on $(g^r,m)$:
	\begin{itemize}
		\item If $i=j$, submits $g^r$ to the Schnorr verifier, get e, et $\H(g^r,m)=e$ and return $e$ to $\A$
		\item if $i \neq j$ returns a random value to $\A$
	\end{itemize}
	\item When $\A$ asks for a signature on $m$, runs Schnorr's simulator to get $(a,e,f)$ and sets $\H(a,m)=e$
	\item When $\A$ outputs a forgery $(m^*,e^*,f^*)$, outputs $f^*$ to the Schnorr verifier 
\end{itemize}


This strategy wins if:
\begin{itemize}
	\item $\A$ indeed produces a forgery ($Pr=\negl$)
	\item $P^*$ did not define $\H(g^r,m)$ before the $j$-th query (with high Pr)
	\item $P^*$ made a correct guess, i.e., $g^{f^*} / h^{e^*} = g^r$ and $m^*=m$ with $(g^r,m)$ as in the $j$-th query ($Pr \ge 1/q$) 
\end{itemize}

Since $P^*$ cannot solve the DL and can only break soundness with negligible probability, $\negl$ must be negligible.























\end{document}